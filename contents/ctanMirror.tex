\section{建立重庆大学CTAN镜像}[Advices on Setting Up a CTAN Mirror]\label{sec:ctanMirror}
\subsection{提议原因}[Reasoning]
\ppt{基础建设}
让用户轻松便捷地下载到\TeX 软件及相关内容是壮大我校\TeX 用户群体的第一步,它的角色如同基础建设一般。

\ppt{CTAN}
作为自由软件,\TeX 的维护工作由全世界的\TeX 爱好者们合力完成,软件分发由CTAN主导完成。CTAN是\href{https://www.ctan.org/}{Comprehensive \TeX Archive Network}的简称,它向全球用户提供\TeX 发行版、宏包、模板、使用文档、源代码、运用程序等文件的下载。

CTAN并不是“一个”网站,它由一个核心站点\footnote{\url{http://dante.ctan.org},由德国\TeX 用户组赞助,Rainer Schoepf主导维护。}和许许多多镜像站点\footnote{站点列表详见\url{https://www.ctan.org/mirrors}。}构成。镜像站点定期与核心站点同步,与核心站点一同向公众提供\TeX 相关资源的下载。使用镜像站点有两个显而易见的优势:

\begin{itemize}
	\item 对于分散在全球各地的用户来说,用户可以选择距离自己最近、网络连通性最好的镜像,从而实现高速下载;
	\item 对于CTAN自身而言,镜像可以为核心站点和其他镜像分担流量,降低整个CTAN网络的压力。
\end{itemize}

\ppt{CTAN:中国}
目前,在中国有4个官方镜像,它们分别是\href{http://mirrors.tuna.tsinghua.edu.cn/CTAN/}{清华大学}、\href{http://mirrors.ustc.edu.cn/CTAN/}{中国科学技术大学}、\href{http://mirrors.hust.edu.cn/CTAN/}{武汉理工大学}、\href{http://mirror.lzu.edu.cn/CTAN/}{兰州大学}。它们的存在使得中国教育网用户和公众网用户享受到了高速便利的\TeX 内容下载。值得注意的是,上述镜像分布于中国北部、中部及西北地区,西南地区还没有镜像。

\ppt{CTAN:重庆大学}
综上所述,倘若我校适配相关资源(我校已有硬件设施,不需要追加投入),进行CTAN的镜像工作,对我校师生乃至中国西南的\TeX 用户来说都是重大利好。

\subsection{依赖条件}[Dependencies]
\ppt{蓝盟}
硬件和技术方面,我校\href{http://lanunion.cqu.edu.cn/}{蓝盟}已经在维护运营重庆大学开源镜像站\footnote{\url{https://mirrors.cqu.edu.cn/}},该站点收录了 CentOS、Archlinux、CentOS、Ubuntu等多个发行版的镜像,另有CPAN等与CTAN结构相似的开源镜像。可以说,我校具备相关硬件和技术条件。

外部支持方面,CTAN一直都欢迎新镜像的加入,在完成相关技术工作后,CTAN会将我校镜像站点加入其官方镜像列表。

\subsection{具体措施}[Detailed Measures]
CTAN提供了\href{https://www.ctan.org/mirrors/register/}{一份详细的操作方案}\footnote{\url{https://www.ctan.org/mirrors/register/}},简略而言,方案包含下列四个步骤:
\begin{enumerate}
	\item 在服务器上提供Web和/或FTP服务;
	\item 使用rsync命令从核心站点拉取数据;
	\item 将rsync命令作为cron job每日执行,保持我校镜像与核心站点同步;
	\item 注册成为官方镜像,加入官方镜像列表。
\end{enumerate}

视我校实际情况,蓝盟方面会按现有条件进行具体操作,望学校给予接洽和支持为宜。蓝盟的兄弟(姐妹),如果你正读到这里,请接受我们的执意和感谢,谢谢你们!