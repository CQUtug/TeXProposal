\subsection{撰写毕业论文对\TeX 的需求}[Dissertation Typesetting with \TeX]\label{sec:texBene}
\ppt{\TeX 排版毕业论文:使用模板的益处}
在\ref{sec:whatistex}节中我们提到,\cquthesis 这一类定制的论文模板可以在很可观的程度上减轻\TeX 用户的排版负担——格式已经预置,标准已经订好,留给用户的事情便是组织内容以及将内容以符合\TeX 及\cquthesis 规范的方式编写出来,排版的任务则留给模板和\TeX 。

进一步地,对于各种用户需求,使用\TeX 对毕业论文进行排版还有以下好处:
\begin{itemize}
	\item 计算机系:代码着色和源程序引用,解决论文中插入的程序的版本之忧;
	\item 外语学院:多国语言文字支持,排版断句以自然段为单位\footnote{Word排版和断句以行为单位。},间距均衡,可读性高;
	\item 理工科:高效、美观、原生支持的数学公式、化学公式排版;
	\item 理工科:基于\texttt{.eps}和\texttt{.pdf}格式的矢量文件支持,矢量插图无限放大不失真;
	\item 更多美妙之处待你去体会和发掘。
\end{itemize}

总之,所有的用户都能花更少的时间排出更优秀的版面,而指导老师和学院也不用再烦心学生论文的格式问题,这便是使用\TeX 和论文模板排版毕业论文的突出优点。

\ppt{\TeX 排版毕业论文:国外高校情况}
在对具代表性五所世界名校(Oxford\cite{o1,o2}, Cambridge\cite{c1,c2}, Harvord\cite{h1,h2}, MIT\cite{m1}, Stanford\cite{s1,s2})的考察中,我们均在其官网上找到校方发布的毕业论文写作的\TeX 模版及\TeX 课程资源\footnote{可点击参考文献编号察看详情。}。值得注意的是,我们在由哈佛大学本科教务主任编写的《荣誉学士学位论文指南》\cite{h1}中找到了这样的内容“\LaTeX{} is recommended but not required.”(“我们推荐学生使用\LaTeX 撰写,不过,这并不是一个强制要求。”)

在以上所有学校中,校方都同时考虑了使用\TeX 及Word的同学的需求。

\ppt{\TeX 排版毕业论文:国内高校情况}
事实上,国内引进\TeX 已有数十年的历史,很多大学的大学生或研究生曾自发制作过自己学校的\TeX 模板。为了初步摸清国内大学\TeX 模板的开发和维护情况,笔者通过搜索引擎调研了国内主要大学(985高校及部分省份省会大学)的\TeX 模板开发情况、维护人员、项目地址、项目热度等信息。
截止2016年6月,国内主要大学(985高校及各省省会大学)的\TeX 模板情况如下\autoref{tab:domestic-statistics}所示:

\begin{longtable}[c]{cccccl}
	\caption{国内主要大学\TeX 模板维护情况统计}\label{tab:domestic-statistics}\\
	\toprule
	\ccell{学校}& \ccell{模板名称}& \ccell{维护人员}& \ccell{项目地址}& \ccell{热度} & \ccell{备注} \\
	\midrule\endfirsthead
	\multicolumn{6}{c}{续表~\thetable\hskip1em 国内主要大学\TeX 模板维护情况统计}\\
	\toprule
	\ccell{学校}& \ccell{模板名称}& \ccell{维护人员}& \ccell{项目地址}& \ccell{热度} & \ccell{备注}\\
	\midrule
	\endhead
	\hline
	\multicolumn{6}{r}{续下页}
	\endfoot
	\endlastfoot
	中国科学院研究生院 & CASthesis & 吴凌云 & \href{http://www.ctex.org/PackageCASthesis}{CTeX页面}\footnote{点击可访问,下同。} & 5 & {\zihao{-5}历史悠久,始祖模板之一} \\
	清华大学 & thuthesis & Ruini Xue & \href{https://ctan.org/pkg/thuthesis}{CTAN页面} & 5 & {\zihao{-5}维护良好,始祖模板之一,用户基数大} \\
	上海交通大学 & SJTUThesis & weijianwen & \href{https://github.com/weijianwen/SJTUThesis}{Github页面} & 5 & {\zihao{-5}维护良好,用户基数大} \\
	南开大学 & NKT & 孙文昌 & \href{http://202.113.29.3/~sunwch/tex/tex.htm}{学校网站} & 4 & {\zihao{-5}数学系教授维护} \\
	电子科技大学 & uestcthesis & Shi Fu­jun& \href{https://www.ctan.org/pkg/uestcthesis}{CTAN页面} & 4 & {\zihao{-5}关注度很不错,研究生院认证,维护略欠缺\footnote{\href{https://www.ctan.org/pkg/uestcthesis}{CTAN版本}最后维护于2013年5月底,好在\href{https://github.com/shifujun/UESTCthesis}{Github版本}维护活跃。学校有\TeX 社区。}} \\
	北京航空航天大学 & BUAAthesis & BHOSC & \href{https://github.com/BHOSC/BUAAthesis}{Github页面} & 4 & {\zihao{-5}} \\
	南京大学 & nju-thesis & Haixing-Hu & \href{https://github.com/Haixing-Hu/nju-thesis}{Github页面} & 4 & {\zihao{-5}历史悠久,最早追溯到6年前,分支众多\footnote{4个分支,现在最热的分支维护得不错,没上CTAN有点可惜。}} \\
	中国科学技术大学 & USTCThesis & ustctug & \href{https://github.com/ustctug/ustcthesis}{Github页面} & 4 & {\zihao{-5}模板代码和运营都很棒\footnote{ustctug是一个很有实力的社团。}} \\
	北京大学 & pkuthss & Casper Ti. Vector & \href{http://ctan.org/pkg/pkuthss}{CTAN页面} & 3 & {\zihao{-5}} \\
	北京师范大学 & BnuThesis & henrysting@gmail.com & \href{http://gerry.lamost.org/blog/?p=811}{个人博客} & 3 & {\zihao{-5}清华系} \\
	山东大学 & sduthesis & Liam Huang & \href{http://www.ctan.org/pkg/sduthesis}{CTAN页面} & 3 & {\zihao{-5}清华系,起步相对早,本硕博支持不全} \\
	湖南大学 & HNUThesis & 杜家宜 & \href{http://www.hnubbs.com/forum.php?mod=viewthread&tid=790603}{学校论坛} & 3 & {\zihao{-5}清华系,不支持本科} \\
	华中科技大学 & hustthesis & hust-latex & \href{https://github.com/hust-latex/hustthesis}{Github页面} & 3 & {\zihao{-5}学校建有CTAN镜像,模板有些瑕疵\footnote{正在谋划提交CTAN.}} \\
	哈尔滨工业大学 & PlutoThesis & dustincys & \href{https://github.com/PlutoThesis}{Github页面} & 3 & {\zihao{-5}历史悠久,分硕博版本和本科版本} \\
	国立台湾大学 & ntu-thesis & tzhuan & \href{https://github.com/tzhuan/ntu-thesis}{Github页面} & 3 & {\zihao{-5}维护良好,用xeCJK解决方案} \\
	武汉大学 & WHUBachelor & 黄正华 & \href{http://aff.whu.edu.cn/huangzh/}{学校网站} & 2 & {\zihao{-5}作者为武大老师,LaTeX经验丰富\footnote{课件使用Beamer来做,模板完成度不高,仅支持本科生。}} \\
	中国海洋大学 & ZwPhdThesis & 周炜 & \href{http://blog.sciencenet.cn/blog-453771-456252.html}{个人博客} & 2 & {\zihao{-5}完成度较低,仅支持博士论文} \\
	西北工业大学 & nwputhesis & Wang Tianshu & \href{http://code.google.com/p/nwputhesis/}{Google Code} & 2 & {\zihao{-5}维护停滞,但是有热度} \\
	东南大学 & SEUThesis & 许元 & \href{http://www.ctan.org/pkg/seuthesis}{CTAN页面} & 2 & {\zihao{-5}本身质量不错,但是没有宣传好和维护好} \\
	华南理工大学 & scutthesis & Alwin Tsui & \href{https://github.com/alwintsui/scutthesis}{Github页面} & 2 & {\zihao{-5}面向Lyx写的一个模板,这点很有意思} \\
	四川大学 & scu\_thesis\_template & cuiao & \href{https://github.com/dahakawang/scu_thesis_template}{Github页面} & 2 & {\zihao{-5}质量一般\footnote{另外有一个,点击访问\href{https://github.com/cuiao/SCU_ThesisDissertation_LaTeXTemplate}{Github页面}}} \\
	中国国防科学技术大学 & nudtpaper & Liu Benyuan & \href{https://github.com/liubenyuan/nudtpaper}{Github页面} & 2 & {\zihao{-5}代码历史悠久,现在的版本是传承下来的} \\
	复旦大学 & FDU-Thesis-Latex & Pandoxie & \href{https://github.com/Pandoxie/FDU-Thesis-Latex}{Github页面} & 2 & {\zihao{-5}上次维护为2年前,硕博双版本} \\
	天津大学 & TJUThesis  & 张井、余蓝涛 & \href{https://github.com/xnth97/TJUThesisLatexTemplate}{Github页面} & 2 & {\zihao{-5}教务处发文介绍推荐} \\
	浙江大学 & write\_with\_LaTeX & Monster, Hamburger & \href{https://github.com/ZJU-Awesome/write_with_LaTeX}{Github页面} & 2 & {\zihao{-5}起步很早,版本很多,代码相对老一些\fancyfoot{,部分分支为清华系。}} \\
	同济大学 & TongjiThesis & gareth@tongji.net & \href{https://sourceforge.net/projects/tongjithesis/}{Sourceforge} & 2 & {\zihao{-5}清华系,最后维护:2009年} \\
	西安交通大学 & XJTUthesis & multiple1902 & \href{https://code.google.com/archive/p/xjtuthesis/}{Google Code} & 1 & {\zihao{-5}起步早,有三个平行项目\footnote{由于学校的支持欠缺,基本上都停止了。这是\href{https://code.google.com/archive/p/xjtuthesis/wikis/Letters.wiki}{一个明显的项目因为缺乏学校支持而失去活力的例子}。}} \\
	中山大学 & sysu\_thesis & 陈颂光 & \href{https://github.com/chungkwong/sysu_thesis}{Github页面} & 1 & {\zihao{-5}代码完成度不错,仅支持本科} \\
	兰州大学 & LZUthesis & mosesnow & \href{https://github.com/mosesnow/LZUthesis}{Github页面} & 1 & {\zihao{-5}中科院系,另外支持Lyx} \\
	东北大学 & NEU\_PHD\_Template & 艾均 & \href{https://github.com/NobodyLiveForever/NEU_PHD_Template}{Github页面} & 1 & {\zihao{-5}上次维护为3年前,仅支持博士} \\
	大连理工大学 & DLUT & whufanwei, yuri\_1985 & \href{https://github.com/whufanwei/}{Github页面} & 1 & {\zihao{-5}上次维护为4年前,分硕博双版本} \\
	厦门大学 & xmu-template & 王玮玮 & \href{https://github.com/wwwxmu/-Latex-}{Github页面} & 1 & {\zihao{-5}模板的代码很简陋\footnote{只涉及了封面和摘要,章节、目录等都没有涉及。厦门大学开有\LaTeX 选修课程。}} \\
	中国人民大学 & ructhesis & ZebinWang & \href{https://github.com/ZebinWang/ructhesis}{Github页面} & 1 & {\zihao{-5}} \\
	华东师范大学 & ecnuthesis & 邓沛 & \href{https://sourceforge.net/projects/ecnuthesis/}{Sourceforge} & 1 & {\zihao{-5}代码思路不错,最后维护:2012年} \\
	重庆大学 & CQUThesis & 李振楠 & \href{https://www.ctan.org/pkg/cquthesis}{CTAN页面} & 1 & {\zihao{-5}刚刚开启,还需要生态建设\footnote{其实,在2014年左右,我校学子陈关才开发过名为CQU的毕业论文模板,托管于\href{https://code.google.com/archive/p/cqu/}{Google Code},不过这个项目已经很久没有维护了。}} \\
	吉林大学 & jluthesis & Zhang Yinhe & \href{https://code.google.com/archive/p/jluthesis/}{Google Code} & 0 & {\zihao{-5}最后维护:2009年} \\
	云南大学 & ynuthesislyx & cherrot & \href{https://github.com/cherrot/ynuthesislyx}{Github页面} & 0 & {\zihao{-5}清华系,基于Lyx,仅本科\footnote{最后维护于3年前。}} \\
	北京理工大学 &  &  &  & 0 & {\zihao{-5}建有CTAN镜像} \\
	中国农业大学 &  &  &  & 0 & {\zihao{-5}有人尝试过} \\
	中央民族大学 &  &  &  & 0 & {\zihao{-5}} \\
	中南大学 &  &  &  & 0 & {\zihao{-5}有人尝试过,但是不成系统} \\
	西北农林科技大学 &  &  &  & 0 & {\zihao{-5}} \\
	广西大学 &  &  &  & 0 & {\zihao{-5}} \\
	香港大学 &  &  &  & 0 & {\zihao{-5}} \\
	澳门大学 &  &  &  & 0 & {\zihao{-5}} \\
	西藏大学 &  &  &  & 0 & {\zihao{-5}} \\
	\bottomrule
\end{longtable}

在解读数据时,以下几点也许需要注意:
\begin{itemize}
	\item “热度”一项由笔者综合评判而得,0为无热度,即认为几乎不会有人使用、讨论、修订、维护相关模板,1$ \sim $5为有热度,数值越大,热度越高,其中规定以清华大学模板\textsc{THUThesis}的热度为5。热度的判定基于以下因素:模板最近维护时间和历史维护频率、关于模板的相关讨论(例如Github Issues)、模板代码质量、模板功能完全程度、模板是否被其他模板所模仿和引用\footnote{清华大学的模板\textsc{THUThesis}、上海交通大学的模板以及哈尔滨工业大学的模板便是有力的例子,很多学校的模板都是在它们的基础上再开发而来。这当中也体现了开源软件的优越性质。}、模板是否模仿和引用其他模板\footnote{例如\cquthesis 即为清华系模板。}、学校支持程度、\TeX 社群维护质量、学校是否维护有CTAN镜像等;
	\item 从模板的托管位置也能看出一些特点,例如:1. 至今仍然托管于Google Code的模板一般都是开发得很早(先驱者),但已经停止维护很久的模板\footnote{Google Code于2015年8月底进入只读模式,如果项目仍然有维护,一般会被迁移到Github。};2. 很多的模板托管于Github,看得出大部分开发者都有此共识:论文模板这一类产品需要和用户保持交流(需求变化、学校规定更新等);3. 托管于CTAN的模板一般都有着不错的质量\footnote{模板上传到CTAN就意味着模板有使用手册,而有使用手册的模板的维护力量一般都不会太差。};4. 有的模板托管于学校相关网站,这一般都会给模板带来不错的关注度。
\end{itemize}


\ppt{\TeX 排版毕业论文:学校支持和认可的必要性}
从以上数据中,我们可以得出一些初步结论:
\begin{enumerate}
	\item \TeX 作为舶来品,其传播和使用与学校当中的科研人员的使用需求相关度高,科研能力强,投稿需求大的学校(在\ref{sec:jounaldemands}节中我们讨论过期刊投稿中使用\TeX 作为文章接受格式的比例和成因),对\TeX 的认知更加全面,从而在学生当中更可能会有愿意和能够进行模板开发的先行者;
	\item 很多历史悠久的模板都没能撑下来,这和学校\TeX 用户稀少,社群凝聚力差,模板维护者放弃维护息息相关;
	\item 几乎没有学校会以官方的身份去开发模板,但是学校可以对模板进行测试、认证以及推荐;
	\item 学校的支持和认可是十分必要的,学校的支持会带来源源不断的\TeX 用户,这使得每年毕业季模板都会有人使用,从而获得持续的反馈;学校的认可则会稳定\TeX 用户们的心,让他们确信,在花功夫习得\TeX 的使用方法后,他们会有用武之地\footnote{西安交通大学的模板开发是\href{https://code.google.com/archive/p/xjtuthesis/wikis/Letters.wiki}{一个明显的项目因为缺乏学校支持而失去活力的遗憾}。}。同时,这也是对模板作者的鼓励和推动。
\end{enumerate}

