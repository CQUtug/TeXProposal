\section{引言}[Introduction]

\ppt{本提案与“双一流”建设之联系}
“必人才日出,然后事业日新;必事业日新,然后生机永畅。”

我国即将实行的“双一流大学”建设计划,是事关未来教育发展的重要改革,其一主要任务——“培养拔尖创新人才。突出人才培养的核心地位,着力培养具有国家使命感和社会责任心,富有创新精神和实践能力的各类创新型、应用型、复合型的优秀人才。”

重庆大学作为国内知名学府,在建设“双一流高校”的历程中,自当紧跟教育改革浪潮,围绕改革目标,通过人才培养与管理机制的创新,努力挖掘学生的的创新潜质,调动创新意识,营造创新氛围,培养一批拥有全球视野、掌握核心竞争力的拔尖创新人才。

\ppt{塑造自由创新的学术写作环境}
从大处着眼——拔尖创新人才的培养,需要充分的资源和自由的环境。从小处着手——学术论文写作是学生学习研究的核心部分,也是创新型人才的基本素养。因此,自由创新的论文写作环境的塑造,是值得提倡的着手点。
第一,为学术论文写作提供全面、自由的支持,将解放学生更多思考及精力投入于学习和研究本身。
第二,写作是学术“日常”,若能使创新能力培养与学术写作糅合,必能有效掀动校园创新之风。
第三,由于历史原因,我国在学术写作工具及资源的多样化支持上,始终未与学术界及西方高校接轨;在国内重点高校中,我校也未能走在前列,这一点亟待我辈追赶。
言而总之,我校在自由创新的学术写作环境的塑造方面,是大有可为的。


\ppt{\TeX 与Word:学术写作}
上世纪 80 年代末 Word 出现以前,排版语言 \TeX 是高校及学术界学术论文写作的绝对主流。Word 出现以后时至今日,\TeX  与 Word 优势互补,成为学术写作界并行的两大标准。然而,由于我国信息革命起步较晚,低门槛的 Word 迅速进入高校成为单一标准,依旧在学术界通行的老传统\TeX 却仅为少数师生所知,其独特功能和魅力没能在国内得以充分体现。这在一定程度上限制了师生在学术写作领域与国际学术界的对接。

\ppt{\TeX:编程思维和创新意识}
除此之外,\TeX 对于学生编程思维和创新意识的培养亦有所增益。这是因为在\TeX 写作过程中,其类似于编程语言的问题解决方式,问题发现和处理的多样化、逻辑性和创造性,使得\TeX 用户在写作过程中有效锻炼了其编程思维、创新意识。

\ppt*{引入\TeX 助力创新型人才培养}
本提案从介绍排版系统\TeX 的背景及特点开始,从研究生期刊论文投稿以及毕业生毕业论文排版工作这两个维度阐述了引入\TeX 作为一种与MS Word平行的写作系统的优势和必要性,最终提出一套基于我校实际情况、有效可行的实施方案。

