\section{结语}[Conclusion]

本提案从介绍排版系统\TeX 的背景和特点开始,从研究生期刊论文投稿以及毕业生毕业论文排版工作这两个维度阐述了引入\TeX 作为一种与Office Word平行的写作系统的优势和必要性,最终提出一套基于我校实际情况、有效可行的实施方案:接受使用\TeX 撰写的毕业论文、开始\TeX 选修课程以及建立重庆大学CTAN镜像。

将\TeX 这个美好的、强大的写作排版工具充分推荐给重大师生,对我们的学术写作水平提高大有裨益,既有利于研究生、博士生的论文投稿工作,又有利于重大全体学子的毕业论文排版工作。长远来看,这对挖掘学生的的创新潜质,调动创新意识,营造创新氛围,培养一批拥有全球视野、掌握核心竞争力的拔尖创新人才极有帮助。

“必人才日出,然后事业日新;必事业日新,然后生机永畅。”

愿母校在“双一流大学”建设计划中走出高度,走出水平。