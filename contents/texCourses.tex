\section{开设\TeX 选修课程}[Advices on Offering \TeX Courses and Resources ]\label{sec:texCourses}
\subsection{提议原因}[Reasoning]

目前,我校的“计算机基础”必修课向全体学生提供了 Office Word 软件操作方法的全面训练,而学校IT中心也出资为全体师生取得了Office Word的集体正版授权,为师生提供了极大便利,这些举措在全国高校中堪称领先。

在前文中,我们陈述了\TeX 在学术写作和出版中的重要性和高效性。将高质量的、高效的资源充分暴露给学习者,使其能够自由选择适合自己的学习与实践方式,才能发挥其积极的挖掘精神,最大化学习与实践效果。因此,我校有必要也有需要面向本科生(毕业论文撰写需求)和研究生(期刊投稿需求)推出\TeX 相关课程培训,这会提高我校师生在学术写作领域的素养,同时对学生创新意识与编程思维的开发培养有利。

再者,学校开设\TeX 相关课程能够大大降低初学者入门难度,降低其学习成本,这对学生掌握知识和撰写论文皆有裨益。

\subsection{依赖条件}[Dependencies]
\begin{enumerate}
	\item 基础建设方面,我校需要拥有自己的CTAN镜像,以便利我校师生获取\TeX 软件和使用文档等相关资源;
	\item 师资方面,我校计算机系和数学系当中应该有一部分教授、讲师为经验丰富的\TeX 用户;
	\item 学校政策方面,接受\TeX 撰写的毕业论文(本科生、研究生、博士生)会增进学生学习\TeX 的动力;
	\item 课程研发方面,可依托重庆大学大学生创新实践中心和重庆大学研究生创新实践基地为实验田(他们也是论文投稿的主力军),逐渐向全校推广。
\end{enumerate}

\subsection{具体措施}[Detailed Measures]
\ppt{硬件建设}
措施可分为硬件建设和软件建设,硬件建设方面:
\begin{itemize}
	\item 完成CTAN镜像搭建(参阅第\ref{sec:ctanMirror}章);
	\item 在向本科生、研究生授课的计算机机房中预装\href{https://en.wikipedia.org/wiki/TeX_Live}{\TeX Live}发行版、\href{http://www.texstudio.org/}{TeX Studio}编辑环境;
\end{itemize}

\ppt{软件建设}
软件建设方面:
\begin{enumerate}
	\item 在校内征集有经验的、对\TeX 有热情的老师;
	\item 为老师提供研讨环境和资源,以方便其制定教案和教材\footnote{可以参考以下资源:
		\href{http://bbs.ctex.org/forum.php?mod=viewthread&tid=68619}{《\LaTeX 排版学习笔记》}、
		\href{http://texdoc.net/texmf-dist/doc/latex/lshort-chinese/lshort-zh-cn.pdf}{《一份不太简短的\LaTeXe 介绍》}、
		\href{http://ptgmedia.pearsoncmg.com/images/9780201362992/samplepages/0201362996.pdf}{《The \LaTeX Companion》}(此处为样张,可在\href{https://www.amazon.com/dp/0201362996}{亚马逊美国}购买)、
		\href{https://www.amazon.cn/dp/B00D1APK0G}{《\LaTeX 入门》}、
		\href{https://www.amazon.cn/dp/B019ERSEAW/}{《21世纪高等院校通识教育规划教材:\LaTeX 科技论文写作简明教程》}。};
	\item 以重庆大学大学生创新实践中心和重庆大学研究生创新实践基地作为试验田进行授课和反馈,可考虑将课程内容与“学术写作”课程联动或合并;
	\item 在教案和教材成熟后,面向全校本科生开设选修课(对毕业论文撰写有利),面向全校理工科研究生开设必修课(对期刊投稿及毕业论文撰写有利)。
\end{enumerate}

需要注意的是,课程应当安排足够的上机课时以及实践型作业。